\documentclass[titlepage]{article}
%\usepackage[french]{babel}
\usepackage[utf8]{inputenc}
\usepackage[T1]{fontenc}
\usepackage{amsmath}
\usepackage{lmodern}
\usepackage{hyperref}
\usepackage{listings} %code
\usepackage{graphicx} %images
\usepackage[export]{adjustbox} %images
\usepackage{comment} %commentaires longs
\excludecomment{toex}
\usepackage{fullpage} %marges
\usepackage[usenames,dvipsnames]{xcolor} %couleurs
%\usepackage{tikz}
%\usepackage{mathrsfs}
%\usetikzlibrary{arrows}

\definecolor{bg}{HTML}{FFFFEE}
\lstset{language=C++,backgroundcolor=\color{bg},frame=single,showspaces=false,basicstyle=\ttfamily,literate={"}{\textquotedbl}1,literate={'}{\textquotesingle}1,keywordstyle=\color{NavyBlue}, breaklines=true,showstringspaces=false,commentstyle=\color{VioletRed},rulecolor=\color{black},otherkeywords={},literate=%
                {é}{{\'e}}{1}%
               {è}{{\`e}}{1}%
               {à}{{\`a}}{1}%
                {ç}{{\c{c}}}{1}%
                {œ}{{\oe}}{1}%
                {ù}{{\`u}}{1}%
                {É}{{\'E}}{1}%
                {È}{{\`E}}{1}%
                {À}{{\`A}}{1}%
                {Ç}{{\c{C}}}{1}%
                {Œ}{{\OE}}{1}%
                {Ê}{{\^E}}{1}%
                {ê}{{\^e}}{1}%
                {î}{{\^i}}{1}%
                {ô}{{\^o}}{1}%
                {û}{{\^u}}{1}%
                {ë}{{\¨{e}}}1
                {û}{{\^{u}}}1
                {â}{{\^{a}}}1
                {Â}{{\^{A}}}1
                {Î}{{\^{I}}}1}


\usepackage{titlesec}
\titleclass{\subsubsubsection}{straight}[\subsection]

\newcounter{subsubsubsection}[subsubsection]
\renewcommand\thesubsubsubsection{\thesubsubsection.\arabic{subsubsubsection}}

\titleformat{\subsubsubsection}
  {\normalfont\normalsize\bfseries}{\thesubsubsubsection}{1em}{}
\titlespacing*{\subsubsubsection}
{0pt}{3.25ex plus 1ex minus .2ex}{1.5ex plus .2ex}
\makeatletter
\def\toclevel@subsubsubsection{4}
\def\l@subsubsubsection{\@dottedtocline{4}{7em}{4em}}
\makeatother
\setcounter{secnumdepth}{4}
\setcounter{tocdepth}{4}
\newcommand{\HRule}{\rule{\linewidth}{0.5mm}}
\definecolor{RoyalRed}{RGB}{157,16, 45}
\titleformat{\section}
%%%%%%%%%%%%%
{\normalfont\LARGE\sffamily}{\underline{Part \thesection :} }{1em}{}
%%%%%%%%%%%%%



%\titleformat*{\section}{\color{RoyalRed}{\thesection}{1em}
%\usepackage{sectsty}
%\sectionfont{\LARGE\color{RoyalRed}\fontfamily{mdugm}}


\begin{document}
\title{LO21 project report: Calculator from the future}
\author{Alexandre Ballet, Anton Ippolitov (GI02)}
\date{P16}
\maketitle

\tableofcontents

\begin{abstract}
In this report, we will present our application developped for the L021 course. A revolutionnary and innovative calculator which will simplify your everyday life.
\end{abstract}

\section{Architecture}

\subsection{QComputer}

Our \textit{QComputer} class is the interface of our application, our \textit{MainWindow}. It sends the content of the \textit{QLineEdit} widget to the \textit{Controleur} instance each time the user presses the \textit{Enter} key or clicks on an operator button. Every instruction is sent to the \textit{parse()} method of the \textit{Controleur} instance by the \textit{QLineEdit} widget. The \textit{QComputer} object refreshes the \textit{QTableWidget} displaying our \textit{Pile} instance each time it is modified.
It also saves/restores the context when the application terminates/launches : the content of the \textit{Pile} instance, the \textit{Variable} and \textit{Programme} objects, and the global settings (keyboard, error sound and number of \textit{Litteral} objects displayed in the \textit{QTableWidget}).\\

\subsection{Controleur}

Our \textit{Controleur} class is a singleton and manages every interaction with the user. It parses and processes each user input : creates the corresponding \textit{Litteral} objects, pushes them into the \textit{Pile} instance, applies the operators, parses and evaluates the \textit{Expression} objects. It also manages the \textit{Memento} class : adds \textit{Memento} states to the \textit{mementoList} stored in the \textit{Pile} instance.\\

We chose to centralise each user input processing into the \textit{parse()} method in order to simplify the use of the \textit{Controleur} class. If the input is not a \textit{Programme} or an \textit{Expression}, which both contain \textit{spaces}, the \textit{parse()} method manually splits the input into seperate words to be processed by the \textit{process()} method. The \textit{Litteral} objects are pushed into the \textit{Pile} instance and the operators are applied.

\subsection{Pile}

As there can only be one \textit{Pile} object, our \textit{Pile} class is a singleton. It has a \textit{QStack} attribute (called \textit{stack}) of pointers to \textit{Litteral} objects.\\

C\texttt{++} code sample:
\begin{lstlisting}
void QComputer::refresh(){
    Pile* pile = Pile::getInstance();

    // the message
    ui->message->setText(pile->getMessage());

    unsigned int nb = 0;
    // delete everything
    for(unsigned int i=0; i<pile->getMaxAffiche(); i++)
        ui->vuePile->item(i,0)->setText("");
    // update
    QStack<Litteral*>::const_iterator it;
    for(it=pile->getIteratorEnd()-1 ; it!=pile->getIteratorBegin()-1 && nb<pile->getMaxAffiche(); nb++, --it){
        ui->vuePile->item(pile->getMaxAffiche()-nb-1,0)->setText((*it)->toString());
    }
}
\end{lstlisting}


\section{LEL}


\end{document}
